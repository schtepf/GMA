\documentclass[a4paper]{article}

\usepackage{vmargin}
\setpapersize[portrait]{A4}
\setmarginsrb{30mm}{10mm}{30mm}{20mm}% left, top, right, bottom
{12pt}{15mm}% head heigth / separation
{0pt}{15mm}% bottom height / separation
%% \setmargnohfrb{30mm}{20mm}{20mm}{20mm}

\setlength{\parindent}{0mm}
\setlength{\parskip}{\medskipamount}

\usepackage[english]{babel}
\usepackage[utf8]{inputenc}

\usepackage[T1]{fontenc}
%% \usepackage{textcomp}  % this can break some outline symbols in CM fonts, use only if absolutely necessary

\usepackage{lmodern}   % type1 computer modern fonts in T1 encoding
%% \usepackage{mathptmx}  % use Adobe Times as standard font with simulated math support
%% \usepackage{mathpazo}  % use Adobe Palatino as standard font with simulated math support

%% \usepackage{pifont}
%% \usepackage{eucal}
\usepackage{mathrsfs} % \mathscr

\usepackage{amsmath,amssymb,amsthm}
\usepackage{graphicx,rotating}
\usepackage{array,hhline,booktabs}
\usepackage{xspace}
\usepackage{url,hyperref}
%% \usepackage{ifthen,calc,hyphenat}
\usepackage{enumitem}
\setlist{noitemsep}

\usepackage{xcolor,tikz}
\usepackage[textwidth=25mm,textsize=small,colorinlistoftodos,backgroundcolor=orange!80]{todonotes} % [disable] to hide all TODOs

\usepackage{natbib}
\bibpunct[:~]{(}{)}{;}{a}{}{,}

%%
%%  INCLUDE: math.tex
%%  
%%  basic mathematical symbols and constructs (not specific to cooccurrences)
%%


%% \setN, \setN[0], \setZ, \setQ, \setR, \setC
%% abbreviations for common number spaces
\newcommand{\setN}[1][]{\mathbb{N}_{#1}} % allows \setN and \setN[0]
\newcommand{\setZ}{\mathbb{Z}}
\newcommand{\setQ}{\mathbb{Q}}
\newcommand{\setR}{\mathbb{R}}
\newcommand{\setC}{\mathbb{C}}

%% \set{el_1, el_2, ...};  \setdef{el}{condition};  \bigset{..}, \bigsetdef{..}{..}
%% extensional and intensional definition of sets, with "big" versions (like \bigl etc.)
\newcommand{\set}[1]{\left\{#1\right\}}
\newcommand{\setdef}[2]{\set{#1\,\left|\,#2\right.}}
\newcommand{\bigset}[1]{\bigl\{#1\bigr\}}
\newcommand{\bigsetdef}[2]{\bigset{#1\bigm|#2}}
\newcommand{\Bigset}[1]{\Bigl\{#1\Bigr\}}
\newcommand{\Bigsetdef}[2]{\Bigset{#1\Bigm|#2}}
\newcommand{\biggset}[1]{\biggl\{#1\biggr\}}
\newcommand{\biggsetdef}[2]{\biggset{#1\biggm|#2}}

%% \compl{X} = complement of set X
\newcommand{\compl}[1]{\mathcal{C} #1}

%% \eps == \epsilon, \si == \sigma, \sisi == \sigma^2, \ka == \kappa
\newcommand{\eps}{\epsilon}
\newcommand{\si}{\sigma}
\newcommand{\sisi}{\sigma^2}
\newcommand{\ka}{\kappa}

%% \abs{expr}, \bigabs{expr}, \norm{expr}, \bignorm{expr}
%% absolute value and norm of expression, with "big" versions
\newcommand{\abs}[1]{\left\lvert#1\right\rvert}
\newcommand{\bigabs}[1]{\bigl\lvert#1\bigr\rvert}
\newcommand{\norm}[1]{\left\lVert#1\right\rVert}
\newcommand{\bignorm}[1]{\bigl\lVert#1\bigr\rVert}

%% \constpi == constant PI (in bold font)
\newcommand{\constpi}{\boldsymbol{\pi}}

%% \dx == "dx";  \dx[z] == "dz";  \dpi == \dx[\pi];  
%% \dG == \dx[G], \dt == \dx[t]
\newcommand{\dx}[1][x]{\,d#1}
\newcommand{\dpi}{\dx[\pi]}
\newcommand{\dG}{\dx[G]}
\newcommand{\dt}{\dx[t]}

%% \Int{\frac{1}{2} x^2}_a^b
%% anti-derivative evaluated to compute definite integral
\newcommand{\Int}[1]{\left[#1\right]}

%% \limdownto{x}{0}
%% limit from above for x -> 0
\newcommand{\limdownto}[2]{\lim_{#1\,\downarrow\,#2}}

%% \iffdef == ":<=>";  \iffdefR == "<=>:"
\newcommand{\iffdef}{\;:\!\iff}
\newcommand{\iffdefR}{\iff\!:\;}

%% \logten(x) 
%% base 10 logarithm, which is always used in the UCS system
\newcommand{\logten}{\log_{10}}

%% \e+3, \e-6, \e-{12}, 5.5\x\e-3
%% engineering-style notation (orders of magnitude) for floating-point numbers
\newcommand{\e}[2]{10^{\ifthenelse{\equal{#1}{+}}{}{#1}#2}}
\newcommand{\x}{\cdot}

%% \Landau{ n^2 }, \bigLandau{ N^2 }
%% Landau symbol ("big oh notation")
\newcommand{\Landau}[1]{\mathcal{O}\left({#1}\right)}
\newcommand{\bigLandau}[1]{\mathcal{O}\bigl({#1}\bigr)}


%%% Local Variables: 
%%% mode: latex
%%% TeX-master: t
%%% End: 

%%
%% some macros for typesetting text
%%

%% \OPEN ... \CLOSE; \OPEN[np] ... \CLOSE[np]
%% bold large brackets for labelled bracketing notation
\newcommand{\OPEN}[1][]{\only{$\boldsymbol{\bigl[}\text{}_{\text{\raisebox{-2pt}{\textsc{#1}}}}$}}
\newcommand{\CLOSE}[1][]{\only{$\text{}_{\text{\raisebox{-2pt}{\textsc{#1}}}}\boldsymbol{\bigr]}$}}

%% \textgap ("_" representing missing letter)
\newcommand{\textgap}{\mbox{\hspace{.4pt}\texttt{\bfseries\secondary{\textunderscore}}\hspace{.4pt}}}

%% -- commands are defined in the hyperref bundle
%% \textstar, \textast (math \star and \ast symbols in text mode, with some extra spacing)
% \newcommand{\textstar}{$\mspace{.8mu}\star\mspace{.8mu}$}
% \newcommand{\textast}{$\ast$}

%% $\p{\ctext{abc}}$ (cited text in mathematical equations, e.g. n-gram probabilities)
\newcommand{\ctext}[1]{\text{\textcite{#1}}}

%% $\p{\btext{abc}}$ (normal black text even in coloured math environment)
\newcommand{\btext}[1]{\text{\foreground{#1}}} 

%% text subscripts and superscripts (can be used in math and text mode)
\newcommand{\tsup}[1]{\ensuremath{^{\text{#1}}}}
\newcommand{\tsub}[1]{\ensuremath{_{\text{#1}}}}

%%% Local Variables: 
%%% mode: latex
%%% TeX-master: ""
%%% End: 

%%
%%  INCLUDE: stat.tex
%%  
%%  symbols and notation for probability theory and statistics
%%


%% \p{X=k};  \pC{X=k}{Y=l};  \pscale{\frac{Z}{S^2}}
%% probability P(X=k), conditional probability P(X=k|Y=l), and variants with scaled parentheses
\newcommand{\p}[1]{\mathop{\mathrm{Pr}}\bigl(#1\bigr)}
\newcommand{\pscale}[1]{\mathop{\mathrm{Pr}}\left(#1\right)}
\newcommand{\pC}[2]{\p{#1\bigm|#2}} 
\newcommand{\pCscale}[2]{\pscale{#1\,\left|\,#2\right.}} 

%% \Exp{X};  \Var{X};  \Exp[0]{X};  \Var[0]{X};  \Expscale{X};  \Varscale{X}
%% expectation E[X] and variance V[X], expectation and variance under null hypothesis, 
%% and variants with scaled brackets
\newcommand{\Exp}[2][]{E_{#1}\bigl[#2\bigr]}
\newcommand{\Var}[2][]{\mathop{\mathrm{Var}}_{#1}\bigl[#2\bigr]}
\newcommand{\Expscale}[2][]{E_{#1}\left[#2\right]}
\newcommand{\Varscale}[2][]{\mathop{\mathrm{Var}}_{#1}\left[#2\right]}

%% \I{f_i = M};  \bigI{\sum_{i=1}^S f_i = N}
%% indicator variable (as in Baayen 2001), and variant with explicitly scaled brackets
\newcommand{\I}[1]{I_{\left[#1\right]}}
\newcommand{\bigI}[1]{I_{\bigl[#1\bigr]}}

%% \Hind;  \Hhom;  \Hnull{\kappa = x}
%% null hypothesis of independence and homogeneity; general null hypothesis identified by condition
\newcommand{\Hind}{H_0}
\newcommand{\Hhom}{H_{0,\, hom}}
\newcommand{\Hnull}[1]{H_{#1}}

%% \confint{\kappa};  \confint[0.99]{\kappa}
%% confidence interval for specified population characteristic (conf. level defaults to \alpha)
\newcommand{\confint}[2][\alpha]{I_{#2,\,#1}}

%% \df = 1
%% degrees of freedom
\newcommand{\df}{\mathit{df}}


%%% Local Variables: 
%%% mode: latex
%%% TeX-master: t
%%% End: 

%%
%% convenience macros for linear algebra (vectors and matrices)
%%

%% \Vector[i]{x} ... vector variable with optional _superscript_ index in parentheses
%% \Vector[']{x} ... special case: ' superscript not enclosed in parentheses
%% \vx, \vy, \vz ... abbreviations for common vector names
\newcommand{\Vector}[2][]{\mathbf{#2}\ifthenelse{\equal{#1}{}}{}{^{(#1)}}}
\newcommand{\vx}[1][]{\Vector[#1]{x}}
\newcommand{\vy}[1][]{\Vector[#1]{y}}
\newcommand{\vz}[1][]{\Vector[#1]{z}}
\newcommand{\vu}[1][]{\Vector[#1]{u}}
\newcommand{\vv}[1][]{\Vector[#1]{v}}
\newcommand{\vw}[1][]{\Vector[#1]{w}}
\newcommand{\vm}[1][]{\Vector[#1]{m}} 
\newcommand{\va}[1][]{\Vector[#1]{a}} % vectors of coefficients
\newcommand{\vb}[1][]{\Vector[#1]{b}} % for basis
\newcommand{\vc}[1][]{\Vector[#1]{c}}
\newcommand{\vd}[1][]{\Vector[#1]{d}}
\newcommand{\ve}[1][]{\Vector[#1]{e}} % for standard basis of R^n
\newcommand{\vn}[1][]{\Vector[#1]{n}} % normal vector
\newcommand{\vnull}[1][]{\Vector[#1]{0}} % neutral element
\newcommand{\vone}[1][]{\Vector[#1]{1}} % vector of ones
\newcommand{\vmu}[1][]{\Vector[#1]{\boldsymbol{\mu}}}

%% \matSigma ... covariance matrix \Sigma (which may need special bold formatting)
\newcommand{\matSigma}{\boldsymbol{\Sigma}}

%% \T ... transpose of a matrix
\newcommand{\T}{^{\text{T}}}

%% \Span{\vb[1],\ldots,\vb[k]} ... span of set of vectors
%% \Rank{...} ... rank of set of vectors or matrix
%% \Det{...}, \det A ... determinant of a set of vectors / a matrix A
%% \Trace{...} ... trace of a matrix A
%% \Image{f}, \Kernel{f} ... image and kernel of a linear map
\newcommand{\Span}[1]{\mathop{\text{sp}}\left(#1\right)}
\newcommand{\Rank}[1]{\mathop{\text{rank}}\left(#1\right)}
\newcommand{\Det}[1]{\mathop{\text{Det}}\left(#1\right)}
%% \det is already defined in the standard library
\newcommand{\Trace}[1]{\mathop{\text{tr}}\left(#1\right)}
\newcommand{\Image}[1]{\mathop{\text{Im}}\left(#1\right)}
\newcommand{\Kernel}[1]{\mathop{\text{Ker}}\left(#1\right)}

%% \dist[2]{\vx}{\vy} ... distance between two vectors (p-metric)
\newcommand{\dist}[3][]{d_{#1}\left(#2, #3\right)}
\newcommand{\bigdist}[3][]{d_{#1}\bigl(#2, #3\bigr)}

%% \sprod{\vu}{\vv} ... scalar product
\newcommand{\sprod}[2]{\left\langle #1, #2 \right\rangle}
\newcommand{\bigsprod}[2]{\bigl\langle #1, #2 \bigr\rangle}


%%% Local Variables: 
%%% mode: latex
%%% TeX-master: ""
%%% End: 


\title{The mathematics of Geometric Multivariate Analysis}
\author{Stephanie Evert}
\date{7 July 2024}

\begin{document}
\maketitle

\tableofcontents

%%%%%%%%%%%%%%%%%%%%%%%%%%%%%%%%%%%%%%%%%%%%%%%%%%%%%%%%%%%%%%%%%%%%%%%%
%%%%%%%%%%%%%%%%%%%%%%%%%%%%%%%%%%%%%%%%%%%%%%%%%%%%%%%%%%%%%%%%%%%%%%%%
\section{Algorithms}
\label{sec:algorithms}

%%%%%%%%%%%%%%%%%%%%%%%%%%%%%%%%%%%%%%%%%%%%%%%%%%%%%%%%%%%%%%%%%%%%%%%%
\subsection{Linear discriminant analysis}
\label{sec:algorithms:lda}

\subsubsection{The LDA algorithm}
\label{sec:algorithms:lda:standard}

\begin{itemize}
\item originally proposed by \citet{Fisher:36} for a one-dimensional discriminant between two groups
  \begin{itemize}
  \item uses $D^2 / S$ as separation criterion where $D$ is the difference between the group means and $S$ the within group variance (computed from within-group covariance matrix $\mathbf{S}$)
  \item directly solves for minimum, resulting in equation system $\mathbf{S} \boldsymbol{\lambda} = \vd$
  \item Fisher does not discuss an extension to multiple groups (using between-group variance as criterion) nor to a multi-dimensional discriminant
  \end{itemize}
\item data matrix $\mathbf{X}\in \setR^{n\times d}$ with $n$ data points $\vx_i\in \setR^d$
\item LDA algorithm as implemented in the \texttt{MASS} package is described by \citet[331--332]{Venables:Ripley:02}:
  \begin{itemize}
  \item matrix of group means $\mathbf{M}\in \setR^{g\times d}$ as row vectors $\vm_j$
  \item group indicator matrix $\mathbf{G}\in \setR^{n\times g}$ with $g_{ij} = 1$ iff $X_i$ belongs to group $j$
  \item $\overline{\vx}\in setR^d$ the overall mean $\overline{\vx} = \frac1n \sum_i \vx_i$
  \item the ``group predictions'' are given by $\mathbf{G}\mathbf{M}$
  \item within-group covariance matrix $\mathbf{W}$ and between-group covariance matrix $\mathbf{B}$ are
    \begin{equation}
      \label{eq:lda:mass-W-B}
      \mathbf{W} = \frac{
        (\mathbf{X} - \mathbf{GM})^T (\mathbf{X} - \mathbf{GM})
      }{ n - g }
      ,\qquad
      \mathbf{B} = \frac{
        (\mathbf{GM} - \vone \overline{\vx}^T)^T (\mathbf{GM} - \vone \overline{\vx}^T)
      }{g - 1}
    \end{equation}
  \item a one-dimensional discriminant is given by a linear combination $\va^T \vx$ that maximises the ratio of between-group to within-group variance along the discriminant axis:
    \begin{equation}
      \label{eq:lda:mass-criterion}
      \frac{\va^T \mathbf{B} \va}{\va^T \mathbf{W} \va}
    \end{equation}
  \item NB: this criterion is proportional to the F-statistic of ANOVA; since it differs only by a fixed factor, the choice of $\va$ also maximises the F-statistic%
    \footnote{See Wikipedia article on \href{https://en.wikipedia.org/wiki/Analysis_of_variance\#The_F-test}{Analysis of variance} for the usual form of the F-statistic. See Wikipedia articles on the \href{https://en.wikipedia.org/wiki/F-test\#Formula_and_calculation}{F-test} and the \href{https://en.wikipedia.org/wiki/F-distribution\#Definition}{F-distribution} for an explanation of the scaling factors involved.}
  \item to find the maximum, compute a sphering $\vy = \mathbf{S} \vx$ of the variables so that the within-group covariance matrix becomes $\mathbf{W}' = \mathbf{I}$
  \item the problem is then to maximise $\va^T \mathbf{B}' \va$ for the transformed between-group matrix $\mathbf{B}$ subject to $\norm{\va} = 1$ (because the transformation $\va' = \mathbf{S}^{-1} \va$ yields the same value for (\ref{eq:lda:mass-criterion}))
  \item $\va$ is then easily found as the largest principal component of $\mathbf{B}'$
  \item for an extension to a multi-dimensional discriminant, the first $r$ principal components can be used, and the number of dimensions can be chosen according to their principal values or $R^2$; while this is plausible in the sphered coordinates, Venables \& Ripley don't explain what separation criterion it optimises in the original coordinate system
  \end{itemize}
\item a different explanation of the LDA algorithm is given by \citet[186--190]{Bishop:06}, who explicitly discusses the extension to multiple classes and a multi-dimensional discriminant \citep[191--192]{Bishop:06}
\item Bishop also points out the problem that it is no longer clear which separation criterion should be maximised and refers to \citet[445--459]{Fukunaga:90} for a detailed exposition of different criteria and their optimisation
\end{itemize}

\paragraph{Useful Wikipedia articles}

\begin{itemize}
\item Analysis of variance: \url{https://en.wikipedia.org/wiki/Analysis_of_variance}
\item F-test: \url{https://en.wikipedia.org/wiki/F-test#Formula_and_calculation}
\item F-distribution: \url{https://en.wikipedia.org/wiki/F-distribution#Definition}
\item MANOVA separation criteria: \url{https://en.wikipedia.org/wiki/Multivariate_analysis_of_variance#Hypothesis_Testing}
\item Linear discriminant analysis: \url{https://en.wikipedia.org/wiki/Linear_discriminant_analysis}, esp.\ \url{https://en.wikipedia.org/wiki/Linear_discriminant_analysis#Multiclass_LDA}
\item Blessing of dimensionality: \url{https://en.wikipedia.org/wiki/Curse_of_dimensionality#Blessing_of_dimensionality} (but more relevant for Azuma paper)
\end{itemize}

\subsubsection{Repeated-measures LDA}
\label{sec:algorithms:lda:repeated}

\begin{itemize}
\item \textbf{repeated-measures} as appropriate terminology: \url{https://en.wikipedia.org/wiki/Repeated_measures_design}
\end{itemize}


%%%%%%%%%%%%%%%%%%%%%%%%%%%%%%%%%%%%%%%%%%%%%%%%%%%%%%%%%%%%%%%%%%%%%%%%
%%%%%%%%%%%%%%%%%%%%%%%%%%%%%%%%%%%%%%%%%%%%%%%%%%%%%%%%%%%%%%%%%%%%%%%%
\section{}
\label{sec:A}

%%%%%%%%%%%%%%%%%%%%%%%%%%%%%%%%%%%%%%%%%%%%%%%%%%%%%%%%%%%%%%%%%%%%%%%%
\subsection{}
\label{sec:A:}


%%%%%%%%%%%%%%%%%%%%%%%%%%%%%%%%%%%%%%%%%%%%%%%%%%%%%%%%%%%%%%%%%%%%%%%%
%%%%%%%%%%%%%%%%%%%%%%%%%%%%%%%%%%%%%%%%%%%%%%%%%%%%%%%%%%%%%%%%%%%%%%%%
\section{}
\label{sec:B}

%%%%%%%%%%%%%%%%%%%%%%%%%%%%%%%%%%%%%%%%%%%%%%%%%%%%%%%%%%%%%%%%%%%%%%%%
\subsection{}
\label{sec:B:}


%%%%%%%%%%%%%%%%%%%%%%%%%%%%%%%%%%%%%%%%%%%%%%%%%%%%%%%%%%%%%%%%%%%%%%%%
%%%%%%%%%%%%%%%%%%%%%%%%%%%%%%%%%%%%%%%%%%%%%%%%%%%%%%%%%%%%%%%%%%%%%%%%
%% \renewcommand{\bibsection}{}    % avoid (or change) section heading 
\bibliographystyle{apalike}
\bibliography{stefan-literature,stefan-publications}  

\newpage
\listoftodos

\end{document}
