\documentclass[a4paper]{article}

\usepackage{vmargin}
\setpapersize[portrait]{A4}
\setmarginsrb{30mm}{10mm}{30mm}{20mm}% left, top, right, bottom
{12pt}{15mm}% head heigth / separation
{0pt}{15mm}% bottom height / separation
%% \setmargnohfrb{30mm}{20mm}{20mm}{20mm}

\setlength{\parindent}{0mm}
\setlength{\parskip}{\medskipamount}

\usepackage[english]{babel}
\usepackage[utf8]{inputenc}

\usepackage[T1]{fontenc}
%% \usepackage{textcomp}  % this can break some outline symbols in CM fonts, use only if absolutely necessary

\usepackage{lmodern}   % type1 computer modern fonts in T1 encoding
%% \usepackage{mathptmx}  % use Adobe Times as standard font with simulated math support
%% \usepackage{mathpazo}  % use Adobe Palatino as standard font with simulated math support

%% \usepackage{pifont}
%% \usepackage{eucal}
\usepackage{mathrsfs} % \mathscr

\usepackage{amsmath,amssymb,amsthm}
\usepackage{graphicx,rotating}
\usepackage{array,hhline,booktabs}
\usepackage{xspace}
\usepackage{url,hyperref}
%% \usepackage{ifthen,calc,hyphenat}
\usepackage{enumitem}
\setlist{noitemsep}

\usepackage{xcolor,tikz}
\usepackage[textwidth=25mm,textsize=footnotesize,colorinlistoftodos,backgroundcolor=orange!80]{todonotes} % [disable] to hide all TODOs

\usepackage{natbib}
\bibpunct[:~]{(}{)}{;}{a}{}{,}

%%
%%  INCLUDE: math.tex
%%  
%%  basic mathematical symbols and constructs (not specific to cooccurrences)
%%


%% \setN, \setN[0], \setZ, \setQ, \setR, \setC
%% abbreviations for common number spaces
\newcommand{\setN}[1][]{\mathbb{N}_{#1}} % allows \setN and \setN[0]
\newcommand{\setZ}{\mathbb{Z}}
\newcommand{\setQ}{\mathbb{Q}}
\newcommand{\setR}{\mathbb{R}}
\newcommand{\setC}{\mathbb{C}}

%% \set{el_1, el_2, ...};  \setdef{el}{condition};  \bigset{..}, \bigsetdef{..}{..}
%% extensional and intensional definition of sets, with "big" versions (like \bigl etc.)
\newcommand{\set}[1]{\left\{#1\right\}}
\newcommand{\setdef}[2]{\set{#1\,\left|\,#2\right.}}
\newcommand{\bigset}[1]{\bigl\{#1\bigr\}}
\newcommand{\bigsetdef}[2]{\bigset{#1\bigm|#2}}
\newcommand{\Bigset}[1]{\Bigl\{#1\Bigr\}}
\newcommand{\Bigsetdef}[2]{\Bigset{#1\Bigm|#2}}
\newcommand{\biggset}[1]{\biggl\{#1\biggr\}}
\newcommand{\biggsetdef}[2]{\biggset{#1\biggm|#2}}

%% \compl{X} = complement of set X
\newcommand{\compl}[1]{\mathcal{C} #1}

%% \eps == \epsilon, \si == \sigma, \sisi == \sigma^2, \ka == \kappa
\newcommand{\eps}{\epsilon}
\newcommand{\si}{\sigma}
\newcommand{\sisi}{\sigma^2}
\newcommand{\ka}{\kappa}

%% \abs{expr}, \bigabs{expr}, \norm{expr}, \bignorm{expr}
%% absolute value and norm of expression, with "big" versions
\newcommand{\abs}[1]{\left\lvert#1\right\rvert}
\newcommand{\bigabs}[1]{\bigl\lvert#1\bigr\rvert}
\newcommand{\norm}[1]{\left\lVert#1\right\rVert}
\newcommand{\bignorm}[1]{\bigl\lVert#1\bigr\rVert}

%% \constpi == constant PI (in bold font)
\newcommand{\constpi}{\boldsymbol{\pi}}

%% \dx == "dx";  \dx[z] == "dz";  \dpi == \dx[\pi];  
%% \dG == \dx[G], \dt == \dx[t]
\newcommand{\dx}[1][x]{\,d#1}
\newcommand{\dpi}{\dx[\pi]}
\newcommand{\dG}{\dx[G]}
\newcommand{\dt}{\dx[t]}

%% \Int{\frac{1}{2} x^2}_a^b
%% anti-derivative evaluated to compute definite integral
\newcommand{\Int}[1]{\left[#1\right]}

%% \limdownto{x}{0}
%% limit from above for x -> 0
\newcommand{\limdownto}[2]{\lim_{#1\,\downarrow\,#2}}

%% \iffdef == ":<=>";  \iffdefR == "<=>:"
\newcommand{\iffdef}{\;:\!\iff}
\newcommand{\iffdefR}{\iff\!:\;}

%% \logten(x) 
%% base 10 logarithm, which is always used in the UCS system
\newcommand{\logten}{\log_{10}}

%% \e+3, \e-6, \e-{12}, 5.5\x\e-3
%% engineering-style notation (orders of magnitude) for floating-point numbers
\newcommand{\e}[2]{10^{\ifthenelse{\equal{#1}{+}}{}{#1}#2}}
\newcommand{\x}{\cdot}

%% \Landau{ n^2 }, \bigLandau{ N^2 }
%% Landau symbol ("big oh notation")
\newcommand{\Landau}[1]{\mathcal{O}\left({#1}\right)}
\newcommand{\bigLandau}[1]{\mathcal{O}\bigl({#1}\bigr)}


%%% Local Variables: 
%%% mode: latex
%%% TeX-master: t
%%% End: 

%%
%% some macros for typesetting text
%%

%% \OPEN ... \CLOSE; \OPEN[np] ... \CLOSE[np]
%% bold large brackets for labelled bracketing notation
\newcommand{\OPEN}[1][]{\only{$\boldsymbol{\bigl[}\text{}_{\text{\raisebox{-2pt}{\textsc{#1}}}}$}}
\newcommand{\CLOSE}[1][]{\only{$\text{}_{\text{\raisebox{-2pt}{\textsc{#1}}}}\boldsymbol{\bigr]}$}}

%% \textgap ("_" representing missing letter)
\newcommand{\textgap}{\mbox{\hspace{.4pt}\texttt{\bfseries\secondary{\textunderscore}}\hspace{.4pt}}}

%% -- commands are defined in the hyperref bundle
%% \textstar, \textast (math \star and \ast symbols in text mode, with some extra spacing)
% \newcommand{\textstar}{$\mspace{.8mu}\star\mspace{.8mu}$}
% \newcommand{\textast}{$\ast$}

%% $\p{\ctext{abc}}$ (cited text in mathematical equations, e.g. n-gram probabilities)
\newcommand{\ctext}[1]{\text{\textcite{#1}}}

%% $\p{\btext{abc}}$ (normal black text even in coloured math environment)
\newcommand{\btext}[1]{\text{\foreground{#1}}} 

%% text subscripts and superscripts (can be used in math and text mode)
\newcommand{\tsup}[1]{\ensuremath{^{\text{#1}}}}
\newcommand{\tsub}[1]{\ensuremath{_{\text{#1}}}}

%%% Local Variables: 
%%% mode: latex
%%% TeX-master: ""
%%% End: 

%%
%%  INCLUDE: stat.tex
%%  
%%  symbols and notation for probability theory and statistics
%%


%% \p{X=k};  \pC{X=k}{Y=l};  \pscale{\frac{Z}{S^2}}
%% probability P(X=k), conditional probability P(X=k|Y=l), and variants with scaled parentheses
\newcommand{\p}[1]{\mathop{\mathrm{Pr}}\bigl(#1\bigr)}
\newcommand{\pscale}[1]{\mathop{\mathrm{Pr}}\left(#1\right)}
\newcommand{\pC}[2]{\p{#1\bigm|#2}} 
\newcommand{\pCscale}[2]{\pscale{#1\,\left|\,#2\right.}} 

%% \Exp{X};  \Var{X};  \Exp[0]{X};  \Var[0]{X};  \Expscale{X};  \Varscale{X}
%% expectation E[X] and variance V[X], expectation and variance under null hypothesis, 
%% and variants with scaled brackets
\newcommand{\Exp}[2][]{E_{#1}\bigl[#2\bigr]}
\newcommand{\Var}[2][]{\mathop{\mathrm{Var}}_{#1}\bigl[#2\bigr]}
\newcommand{\Expscale}[2][]{E_{#1}\left[#2\right]}
\newcommand{\Varscale}[2][]{\mathop{\mathrm{Var}}_{#1}\left[#2\right]}

%% \I{f_i = M};  \bigI{\sum_{i=1}^S f_i = N}
%% indicator variable (as in Baayen 2001), and variant with explicitly scaled brackets
\newcommand{\I}[1]{I_{\left[#1\right]}}
\newcommand{\bigI}[1]{I_{\bigl[#1\bigr]}}

%% \Hind;  \Hhom;  \Hnull{\kappa = x}
%% null hypothesis of independence and homogeneity; general null hypothesis identified by condition
\newcommand{\Hind}{H_0}
\newcommand{\Hhom}{H_{0,\, hom}}
\newcommand{\Hnull}[1]{H_{#1}}

%% \confint{\kappa};  \confint[0.99]{\kappa}
%% confidence interval for specified population characteristic (conf. level defaults to \alpha)
\newcommand{\confint}[2][\alpha]{I_{#2,\,#1}}

%% \df = 1
%% degrees of freedom
\newcommand{\df}{\mathit{df}}


%%% Local Variables: 
%%% mode: latex
%%% TeX-master: t
%%% End: 

%%
%% convenience macros for linear algebra (vectors and matrices)
%%

%% \Vector[i]{x} ... vector variable with optional _superscript_ index in parentheses
%% \Vector[']{x} ... special case: ' superscript not enclosed in parentheses
%% \vx, \vy, \vz ... abbreviations for common vector names
\newcommand{\Vector}[2][]{\mathbf{#2}\ifthenelse{\equal{#1}{}}{}{^{(#1)}}}
\newcommand{\vx}[1][]{\Vector[#1]{x}}
\newcommand{\vy}[1][]{\Vector[#1]{y}}
\newcommand{\vz}[1][]{\Vector[#1]{z}}
\newcommand{\vu}[1][]{\Vector[#1]{u}}
\newcommand{\vv}[1][]{\Vector[#1]{v}}
\newcommand{\vw}[1][]{\Vector[#1]{w}}
\newcommand{\vm}[1][]{\Vector[#1]{m}} 
\newcommand{\va}[1][]{\Vector[#1]{a}} % vectors of coefficients
\newcommand{\vb}[1][]{\Vector[#1]{b}} % for basis
\newcommand{\vc}[1][]{\Vector[#1]{c}}
\newcommand{\vd}[1][]{\Vector[#1]{d}}
\newcommand{\ve}[1][]{\Vector[#1]{e}} % for standard basis of R^n
\newcommand{\vn}[1][]{\Vector[#1]{n}} % normal vector
\newcommand{\vnull}[1][]{\Vector[#1]{0}} % neutral element
\newcommand{\vone}[1][]{\Vector[#1]{1}} % vector of ones
\newcommand{\vmu}[1][]{\Vector[#1]{\boldsymbol{\mu}}}

%% \matSigma ... covariance matrix \Sigma (which may need special bold formatting)
\newcommand{\matSigma}{\boldsymbol{\Sigma}}

%% \T ... transpose of a matrix
\newcommand{\T}{^{\text{T}}}

%% \Span{\vb[1],\ldots,\vb[k]} ... span of set of vectors
%% \Rank{...} ... rank of set of vectors or matrix
%% \Det{...}, \det A ... determinant of a set of vectors / a matrix A
%% \Trace{...} ... trace of a matrix A
%% \Image{f}, \Kernel{f} ... image and kernel of a linear map
\newcommand{\Span}[1]{\mathop{\text{sp}}\left(#1\right)}
\newcommand{\Rank}[1]{\mathop{\text{rank}}\left(#1\right)}
\newcommand{\Det}[1]{\mathop{\text{Det}}\left(#1\right)}
%% \det is already defined in the standard library
\newcommand{\Trace}[1]{\mathop{\text{tr}}\left(#1\right)}
\newcommand{\Image}[1]{\mathop{\text{Im}}\left(#1\right)}
\newcommand{\Kernel}[1]{\mathop{\text{Ker}}\left(#1\right)}

%% \dist[2]{\vx}{\vy} ... distance between two vectors (p-metric)
\newcommand{\dist}[3][]{d_{#1}\left(#2, #3\right)}
\newcommand{\bigdist}[3][]{d_{#1}\bigl(#2, #3\bigr)}

%% \sprod{\vu}{\vv} ... scalar product
\newcommand{\sprod}[2]{\left\langle #1, #2 \right\rangle}
\newcommand{\bigsprod}[2]{\bigl\langle #1, #2 \bigr\rangle}


%%% Local Variables: 
%%% mode: latex
%%% TeX-master: ""
%%% End: 


\title{The mathematics of Geometric Multivariate Analysis}
\author{Stephanie Evert}
\date{7 July 2024}

\begin{document}
\maketitle

\tableofcontents

%%%%%%%%%%%%%%%%%%%%%%%%%%%%%%%%%%%%%%%%%%%%%%%%%%%%%%%%%%%%%%%%%%%%%%%%
%%%%%%%%%%%%%%%%%%%%%%%%%%%%%%%%%%%%%%%%%%%%%%%%%%%%%%%%%%%%%%%%%%%%%%%%
\section{Linear discriminant analysis}
\label{sec:lda}

%%%%%%%%%%%%%%%%%%%%%%%%%%%%%%%%%%%%%%%%%%%%%%%%%%%%%%%%%%%%%%%%%%%%%%%%
\subsection{Background material}
\label{sec:lda:background}

\begin{itemize}
\item originally proposed by \citet{Fisher:36} for a one-dimensional discriminant between two groups
  \begin{itemize}
  \item uses $D^2 / S$ as separation criterion where $D$ is the difference between the group means and $S$ the within group variance (computed from within-group covariance matrix $\mathbf{S}$)
  \item directly solves for minimum, resulting in equation system $\mathbf{S} \boldsymbol{\lambda} = \vd$
  \item Fisher does not discuss an extension to multiple groups (using between-group variance as criterion) nor to a multi-dimensional discriminant
  \end{itemize}
\item data matrix $\mathbf{X}\in \setR^{n\times d}$ with $n$ data points $\vx_i\in \setR^d$
\item LDA algorithm as implemented in the \texttt{MASS} package is described by \citet[331--332]{Venables:Ripley:02}:
  \begin{itemize}
  \item matrix of group means $\mathbf{M}\in \setR^{g\times d}$ as row vectors $\vm_j$
  \item group indicator matrix $\mathbf{G}\in \setR^{n\times g}$ with $g_{ij} = 1$ iff $X_i$ belongs to group $j$
  \item $\overline{\vx}\in \setR^d$ the overall mean $\overline{\vx} = \frac1n \sum_i \vx_i$
  \item the ``group predictions'' are given by $\mathbf{G}\mathbf{M}$
  \item within-group covariance matrix $\mathbf{W}$ and between-group covariance matrix $\mathbf{B}$ are
    \begin{equation}
      \label{eq:lda:mass-W-B}
      \mathbf{W} = \frac{
        (\mathbf{X} - \mathbf{GM})\T (\mathbf{X} - \mathbf{GM})
      }{ n - g }
      ,\qquad
      \mathbf{B} = \frac{
        (\mathbf{GM} - \vone \overline{\vx}\T)\T (\mathbf{GM} - \vone \overline{\vx}\T)
      }{g - 1}
    \end{equation}
  \item a one-dimensional discriminant is given by a linear combination $\va\T \vx$ that maximises the ratio of between-group to within-group variance along the discriminant axis:
    \begin{equation}
      \label{eq:lda:mass-criterion}
      \frac{\va\T \mathbf{B} \va}{\va\T \mathbf{W} \va}
    \end{equation}
  \item NB: this criterion is proportional to the F-statistic of ANOVA; since it differs only by a fixed factor, the choice of $\va$ also maximises the F-statistic%
    \footnote{See Wikipedia article on \href{https://en.wikipedia.org/wiki/Analysis_of_variance\#The_F-test}{Analysis of variance} for the usual form of the F-statistic. See Wikipedia articles on the \href{https://en.wikipedia.org/wiki/F-test\#Formula_and_calculation}{F-test} and the \href{https://en.wikipedia.org/wiki/F-distribution\#Definition}{F-distribution} for an explanation of the scaling factors involved.}
  \item to find the maximum, compute a sphering $\vy = \mathbf{S} \vx$ of the variables so that the within-group covariance matrix becomes $\mathbf{W}' = \mathbf{I}$
  \item the problem is then to maximise $\va\T \mathbf{B}' \va$ for the transformed between-group matrix $\mathbf{B}$ subject to $\norm{\va} = 1$ (because the transformation $\va' = \mathbf{S}^{-1} \va$ yields the same value for (\ref{eq:lda:mass-criterion}))
  \item $\va$ is then easily found as the largest principal component of $\mathbf{B}'$
  \item for an extension to a multi-dimensional discriminant, the first $r$ principal components can be used, and the number of dimensions can be chosen according to their principal values or $R^2$; while this is plausible in the sphered coordinates, Venables \& Ripley don't explain what separation criterion it optimises in the original coordinate system
  \end{itemize}
\item a different explanation of the LDA algorithm is given by \citet[186--190]{Bishop:06}, who explicitly discusses the extension to multiple classes and a multi-dimensional discriminant \citep[191--192]{Bishop:06}
\item Bishop also points out the problem that it is no longer clear which separation criterion should be maximised and refers to \citet[445--459]{Fukunaga:90} for a detailed exposition of different criteria and their optimisation
\end{itemize}

\paragraph{Useful Wikipedia articles}

\begin{itemize}
\item Analysis of variance: \url{https://en.wikipedia.org/wiki/Analysis_of_variance}
\item F-test: \url{https://en.wikipedia.org/wiki/F-test#Formula_and_calculation}
\item F-distribution: \url{https://en.wikipedia.org/wiki/F-distribution#Definition}
\item MANOVA separation criteria: \url{https://en.wikipedia.org/wiki/Multivariate_analysis_of_variance#Hypothesis_Testing}
\item Linear discriminant analysis: \url{https://en.wikipedia.org/wiki/Linear_discriminant_analysis}, esp.\ \url{https://en.wikipedia.org/wiki/Linear_discriminant_analysis#Multiclass_LDA}
\item Blessing of dimensionality: \url{https://en.wikipedia.org/wiki/Curse_of_dimensionality#Blessing_of_dimensionality} (but more relevant for Azuma paper)
\end{itemize}

\paragraph{Other material}

\begin{itemize}
\item Implementation of \texttt{lda()} in \url{https://github.com/cran/MASS/blob/master/R/lda.R}%
  \footnote{local copy in \url{file:///Users/ex47emin/Software/R/MASS-GIT/R/lda.R}}
\end{itemize}


%%%%%%%%%%%%%%%%%%%%%%%%%%%%%%%%%%%%%%%%%%%%%%%%%%%%%%%%%%%%%%%%%%%%%%%%
\subsection{Analysis of variance}
\label{sec:lda:anova}

Unsurprisingly, LDA \citep{Fisher:36} is closely connected to the analysis of variance or \textbf{ANOVA} \citep{Fisher:25}. We start by summarising the ANOVA method following the exposition in \citet[754--761]{DeGroot:Schervish:12}, but with modified notation.

\begin{itemize}
\item data: $n$ observations $y_i\in \setR$ belonging to $g$ groups; $g_i\in \set{1,\ldots,g}$ indicates group membership of $y_i$; group sizes are given by $n_j = \abs{\set{g_i = j}} = \sum_{g_i = j} 1$
\item assumptions: items of group $j$ are i.i.d.\ samples from normal distribution $N(\mu_j, \sigma^2)$; variance $\sigma^2$ is equal for all groups, but the group means $\mu_j$ may be different
\item ANOVA null hypothesis to be tested is $H_0: \mu_1 = \ldots = \mu_g$ (equal group means)
\item observed overall mean $m$ and group means $m_j$ are given by
  \begin{equation}
    \label{eq:lda:anova:means}
    m = \frac1n \sum_{i=1}^n y_i \qquad
    m_j = \frac1{n_j} \sum_{g_i = j} y_i
  \end{equation}
\item basic idea: \textbf{sum of squares} as measure of variability of the data set can be partitioned into within-group and between-group components: $S^2 = S^2_W + S^2_B$ \citep[758]{DeGroot:Schervish:12}
  \begin{align*}
    S^2 &= \sum_{i=1}^n (y_i - m)^2 \\
    S^2_W &= \sum_{j=1}^g \sum_{g_i = j} (y_i - m_j)^2 = \sum_{i=1}^n (y_i - m_{g_i})^2 \\
    S^2_B &= \sum_{j=1}^g n_j (m_j - m)^2 = \sum_{i=1}^n (m_{g_i} - m)^2
  \end{align*}
\item $S^2_W / \sigma^2$ has a $\chi^2_{n-g}$ distribution \citep[757]{DeGroot:Schervish:12}; it follows that the \textbf{within-group variance} $W$ is an unbiased estimator of $\sigma^2$
  \begin{equation}
    \label{eq:lda:anova:W}
    W = \frac{ \sum_{i=1}^n (y_i - m_{g_i})^2 }{ n - g }
  \end{equation}
\item under $H_0$ it can be shown that $S^2_B / \sigma^2$ has a $\chi^2_{g-1}$ distribution \citep[759]{DeGroot:Schervish:12}%
  \footnote{note that under $H_0$ we have $m_j \sim N(\mu, \sigma^2 / n_j)$} %
  and the \textbf{between-group variance} $B$ is also an unbiased estimator of $\sigma^2$
  \begin{equation}
    \label{eq:lda:anova:B}
    B = \frac{ \sum_{j=1}^g n_j (m_j - m)^2 }{ g - 1 }
  \end{equation}
\item if $H_0$ does not hold, we expect $B$ to be larger than $\sigma^2$ (because of the added variability between the group means $\mu_j$) so that the ratio
  \begin{equation}
    \label{eq:lda:anova:U}
    F = \frac{B}{W} = \frac{S^2_B / (g - 1)}{S^2_W / (n - g)}
  \end{equation}
  is a suitable test statistic for ANOVA; p-values can be obtained from its $F_{g-1, n-g}$ distribution under $H_0$ \citep[759]{DeGroot:Schervish:12}
\end{itemize}

Analysis of variance can be generalised to a comparison of group means for multivariate data (\textbf{MANOVA}). Many concepts carry over in a straightforward way, but a suitable test statistic and its sampling distribution under $H_0$ are less obvious. The summary shown here is based on the Wikipedia article \href{https://en.wikipedia.org/wiki/Multivariate_analysis_of_variance}{\emph{Multivariate analysis of variance}}, again with modified notation.

\begin{itemize}
\item data are vectors $\vy_i\in \setR^d$ with group membership $g_i$
\item assumption: each group $j$ has a multivariate normal distribution $N(\vmu_j, \matSigma)$ with equal covariance matrix $\matSigma$, but possibly different group means $\vmu_j$
\item MANOVA null hypothesis $H_0: \vmu_1 = \ldots = \vmu_g$
\item overall mean $\vm$ and group means $\vm_j$ are
  \begin{equation}
    \label{eq:lda:manova:means}
    \vm = \frac1n \sum_{i=1}^n \vy_i \qquad
    \vm_j = \frac1{n_j} \sum_{g_i = j} \vy_i
  \end{equation}
\item instead of a sum of squares, we partition the \textbf{covariance matrix} $\mathbf{C}$ given by
  \begin{equation}
    \label{eq:lda:manova:C}
    \mathbf{C} = \frac1{n-1} \sum_{i=1}^n (\vy_i - \vm) (\vy_i - \vm)\T
  \end{equation}
  where the transpose cross-product computes all squares and products of $\vy_i - \vm$
\item we partition $\mathbf{C}$ into within-group and between-group covariance matrices in the form
  \[
    (n-1) \mathbf{C} = (n - g) \mathbf{W} + (g - 1) \mathbf{B}
  \]
  with
  \begin{align}
    \label{eq:lda:manova:W}
    \mathbf{W} &= \frac1{n - g} \sum_{i=1}^n (\vy_i - \vm_{g_i}) (\vy_i - \vm_{g_i})\T \\
    \label{eq:lda:manova:B}
    \mathbf{B} &= \frac1{g - 1} \sum_{j=1}^g n_j (\vm_j - \vm) (\vm_j - \vm)\T
  \end{align}
  \citep[cf.][191--192]{Bishop:06}
\item according to the Wikipedia article \href{https://en.wikipedia.org/wiki/Multivariate_normal_distribution\#Parameter_estimation}{\emph{Multivariate normal distribution}}\footnote{but [\emph{citation needed}]} $\mathbf{C}$ is an unbiased estimator of $\matSigma$ under $H_0$; correspondingly, $\mathbf{W}$ is always an unbiased estimator of $\matSigma$ and $\mathbf{B}$ is under $H_0$
\item this motivates $\mathbf{A} = \mathbf{B} \mathbf{W}^{-1}$ as a widely-used test criterion with $\mathbf{A}\approx \mathbf{I}$ under $H_0$; intuitively, $\mathbf{A}$ compares the shape and magnitude of the between-group covariance matrix against the within-group covariance matrix; it should, in particular, also detected cases where there are unexpectedly large differences between group means along an axis that has small within-group variance
\item the precise choice of a test statistic is less obvious; common options include Wilks's lambda $\lambda_{\text{Wilks}} = \Det{\mathbf{I} + \mathbf{A}}^{-1}$ and the Lawley-Hotelling trace $\lambda_{\text{LH}} = \Trace{\mathbf{A}}$
\item exact distributions of these test statistics under $H_0$ are not available, except for $g = 2$, where they reduce to Hotelling's $t^2$ distribution\footnote{but [\emph{citation needed}]}
\end{itemize}

%%%%%%%%%%%%%%%%%%%%%%%%%%%%%%%%%%%%%%%%%%%%%%%%%%%%%%%%%%%%%%%%%%%%%%%%
\subsection{The LDA algorithm}
\label{sec:lda:standard}

\subsubsection{Data set and goals of LDA}
\label{sec:lda:standard:goals}

\begin{itemize}
\item data are $n$ feature vectors $\vx_i \in \setR^d$ combined into a data matrix $\mathbf{X}\in \setR^{n\times d}$
\item each data point is assigned to one of $g$ groups indicated by $g_i\in \set{1, \ldots, g}$; the sizes of the groups are $n_j = \abs{\set{g_i = j}}$
\item LDA aims to find a one-dimensional projection (the \textbf{discriminant}) that maximises the separation between groups
\item \citet{Fisher:36} and most textbooks introduce LDA for the special case $g = 2$ of two groups, for which an optimal discriminant can easily be derived; we formulate its generalisation to an arbitrary number of groups based on the $F$ statistic of ANOVA%
  \footnote{our approach implicitly builds on the same distributional assumptions as ANOVA, which motivate the use of the $F$ statistic as an optimality criterion; they are not a necessary pre-requisite for application of the LDA method, but results will be most sensible if $\matSigma$ is roughly equal across all groups}
\item \textbf{task}: find axis $\va\in \setR^d$ that maximises the $F$ statistic of discriminant scores $y_i = \va\T \vx_i$
\end{itemize}

\subsubsection{Covariance matrix and projection}
\label{sec:lda:standard:covmat}

\begin{itemize}
\item this more explicit derivation corresponds to the LDA algorithm described by \citet[331--332]{Venables:Ripley:02} and thus to (one variant of) its implementation in the MASS package
\item overall mean $\vm$ and group means $\vm_j$ are given by
  \begin{equation}
    \label{eq:lda:means}
    \vm = \frac1n \sum_{i=1}^n \vx_i \qquad
    \vm_j = \frac1{n_j} \sum_{g_i = j} \vx_i
  \end{equation}
\item within-group and between-group \textbf{covariance matrices} are defined as in (\ref{eq:lda:manova:W}) and (\ref{eq:lda:manova:B})
  \begin{align}
    \label{eq:lda:W}
    \mathbf{W} &= \frac1{n - g} \sum_{i=1}^n (\vx_i - \vm_{g_i}) (\vx_i - \vm_{g_i})\T \\
    \label{eq:lda:B}
    \mathbf{B} &= \frac1{g - 1} \sum_{j=1}^g n_j (\vm_j - \vm) (\vm_j - \vm)\T
  \end{align}
\item given an axis $\va\in \setR^d$, the one-dimensional discriminant scores of data points are $y_i = \va\T \vx_i$; due to linearity the overall and group means are $m = \va\T \vm$ and $m_j = \va\T \vm_j$
\item hence the within-group variance (\ref{eq:lda:anova:W}) can be computed as
  \begin{equation}
    \label{eq:lda:discW}
    \begin{split}
      W &= \frac1{n - g} \sum_{i=1}^n (\va\T \vx_i - \va\T \vm_{g_i})^2 \\
        &= \frac1{n - g} \sum_{i=1}^n (\va\T \vx_i - \va\T \vm_{g_i}) (\va\T \vx_i - \va\T \vm_{g_i})\T \\
        &= \frac1{n - g} \sum_{i=1}^n \va\T (\vx_i - \vm_{g_i}) (\vx_i - \vm_{g_i})\T \va \\
        &= \va\T \mathbf{W} \va
    \end{split}
  \end{equation}
\item analogously the between-group variance (\ref{eq:lda:anova:B}) can be computed as
  \begin{equation}
    \label{eq:lda:discB}
    B = \va\T \mathbf{B} \va
  \end{equation}
\item our goal is to find an axis $\va$ that maximises the test statistic $F = B / W$, so that we can most clearly reject $H_0$ of equal group means for the discriminant scores $y_i$
  \begin{equation}
    \label{eq:lda:F-stat}
    F = \frac{B}{W} = \frac{ \va\T \mathbf{B} \va }{ \va\T \mathbf{W} \va }
  \end{equation}
\end{itemize}

\subsubsection{Coordinate transformation}
\label{sec:lda:standard:sphering}

\begin{itemize}
\item a convenient approach starts by \textbf{sphering} the within-group covariance matrix $\mathbf{W}$ with a coordinate transformation $\vx' = \mathbf{S} \vx$ such that in the new coordinate system $\mathbf{W}' = \mathbf{I}$
\item the homomorphism preserves overall and group means: $\vm' = \mathbf{S} \vm$ and $\vm'_j = \mathbf{S} \vm_j$
\item the within-group covariance matrix $\mathbf{W}'$ in the new coordinate system is
  \begin{equation}
    \label{eq:lda:Wprime}
    \begin{split}
      \mathbf{W}'
      &= \frac1{n-g} \sum_{i=1}^n (\vx'_i - \vm'_{g_i}) (\vx'_i - \vm'_{g_i})\T \\
      &= \frac1{n-g} \sum_{i=1}^n (\mathbf{S}\vx_i - \mathbf{S}\vm_{g_i}) (\mathbf{S}\vx_i - \mathbf{S}\vm_{g_i})\T \\
      &= \mathbf{S} \mathbf{W} \mathbf{S}\T
    \end{split}
  \end{equation}
\item in the same way we can easily see that the between-group covariance matrix is $\mathbf{B}' = \mathbf{S} \mathbf{B} \mathbf{S}\T$
\item a suitable coordinate transformation $\mathbf{S}$ can be derived from the \textbf{eigenvalue decomposition} of the symmetric, positive semidefinite matrix $\mathbf{W} = \mathbf{U} \mathbf{D} \mathbf{U}\T$ where $\mathbf{D}$ is the diagonal matrix of eigenvalues $\lambda_1 \geq \lambda_2 \geq \ldots \geq \lambda_d$ and the columns of $\mathbf{U}$ are the corresponding eigenvectors; note that $\mathbf{U}$ is an orthonormal matrix, i.e.\ $\mathbf{U}^{-1} = \mathbf{U}\T$ or $\mathbf{U} \mathbf{U}\T = \mathbf{U}\T \mathbf{U} = \mathbf{I}$
\item prerequisite: $\mathbf{W}$ must be positive definite ($\lambda_d > 0$) with good condition number $\lambda_1 / \lambda_d$
\item then we can define $\mathbf{S} = \mathbf{D}^{-\frac12} \mathbf{U}\T$ with inverse transformation $\mathbf{S}^{-1} = \mathbf{U} \mathbf{D}^{\frac12}$
\item within-group covariance matrix $\mathbf{W}'$ in the transformed coordinates:
  \begin{equation}
    \label{eq:lda:Wprime-I}
    \mathbf{W}'
    = \mathbf{S} \mathbf{W} \mathbf{S}\T
    = \mathbf{D}^{-\frac12} \mathbf{U}\T ( \mathbf{U} \mathbf{D} \mathbf{U}\T ) \mathbf{U} \mathbf{D}^{-\frac12} 
    = \mathbf{D}^{-\frac12} \mathbf{D} \mathbf{D}^{-\frac12}
    = \mathbf{I}
  \end{equation}
\end{itemize}

\subsubsection{LDA discriminant}
\label{sec:lda:standard:discriminant}

\begin{itemize}
\item since the discriminant axis $\va$ describes a linear form $\vx \mapsto y = \va\T \vx$ it is subjected to the inverse transformation $(\va')\T = \va\T \mathbf{S}^{-1}$, which corresponds to the identity $\va = \mathbf{S}\T \va'$
\item confirm that the F-statistic is invariant under these transformations:
  \begin{equation}
    \label{eq:lda:F-transformation}
    F = \frac{B}{W} = \frac{ \va\T \mathbf{B} \va }{ \va\T \mathbf{W} \va }
    = \frac{ (\va')\T \mathbf{S} \mathbf{B} \mathbf{S}\T \va' }{ (\va')\T \mathbf{S} \mathbf{W} \mathbf{S}\T \va' }
    = \frac{ (\va')\T \mathbf{B}' \va' }{ (\va')\T \mathbf{W}' \va' }
    = \frac{B'}{W'}
  \end{equation}
\item it is thus sufficient to find $\va'$ that maximises $F$ in the transformed coordinates:
  \begin{equation}
    \label{eq:lda:Fprime}
    \frac{B'}{W'}
    = \frac{ (\va')\T \mathbf{B}' \va' }{ (\va')\T \mathbf{W}' \va' }
    = \frac{ (\va')\T \mathbf{B}' \va' }{ (\va')\T \va' }
    = \frac{ (\va')\T \mathbf{B}' \va' }{ \norm{\va'}^2 }    
  \end{equation}
  or equivalently maximise $(\va')\T \mathbf{B}' \va'$ under the constraint $\norm{\va'} = 1$
\item it is well-known that the solution is given by the first eigenvector $\vv_1$ of $\mathbf{B'}$; this is also easy to see: for every eigenvector $\vv_i$ we have $\norm{\vv_i} = 1$ and $\vv_i\T \mathbf{B}' \vv_i = \mu_i$ the corresponding eigenvalue, so the best choice is $\va' = \vv_1$ with the largest eigenvalue $\mu_1$
\item the optimal discriminant axis in original coordinates is thus $\va = \mathbf{S}\T \vv_1$
\end{itemize}

\subsubsection{LDA with multiple discriminants}
\label{sec:lda:standard:multiple}

\begin{itemize}
\item for $g > 2$ it is usually necessary to consider a multi-dimensional \textbf{discriminant space} (of up to $g - 1$ dimensions) to achieve an optimal separation of groups
\item we thus have multiple discriminants $\va_1, \ldots, \va_r\in \setR^d$ describing linear forms $\vx \mapsto y_k = \va_k\T\vx$, which we collect as rows of the \textbf{discriminant matrix} $\mathbf{A}\in \setR^{r\times d}$, so that $\vy = \mathbf{A} \vx \in \setR^r$
\item overall and group means in the \textbf{discriminant space} are $\tvm = \mathbf{A} \vm$ and $\tvm_j = \mathbf{A} \vm_j$ (due to linearity); within-group and between-group covariance matrices are obtained in analogy to (\ref{eq:lda:discW}) and (\ref{eq:lda:discB}) as
  \begin{equation}
    \label{eq:lda:tildeWB}
    \tmat W = \mathbf{A} \mathbf{W} \mathbf{A}\T, \qquad
    \tmat B = \mathbf{A} \mathbf{B} \mathbf{A}\T
  \end{equation}
\item for measuring separation of groups within the discriminant space we use the Lawley-Hotelling trace as a MANOVA test statistic:
  \begin{equation}
    \label{eq:lda:LH}
    \llh(\mathbf{A}) = \Trace{\tmat B \tmat W^{-1}}
  \end{equation}
   our goal is to find a discriminant matrix $\mathbf{A}$ that maximises $\llh(\mathbf{A})$
 \item a first important property of $\llh$ is its invariance under coordinate transformations in the discriminant space; for any coordinate transformation $\mathbf{S}\in \setR^{r\times r}$ we have in analogy to (\ref{eq:lda:Wprime})
   \begin{equation}
     \label{eq:lda:tildeWBtransform}
     \tmat B\mapsto \mathbf{S} \tmat B \mathbf{S}\T,\qquad
     \tmat W^{-1} \mapsto (\mathbf{S} \tmat W \mathbf{S}\T)^{-1}
     = (\mathbf{S}\T)^{-1} \tmat W^{-1} \mathbf{S}^{-1}
   \end{equation}
   and hence
   \begin{equation}
     \label{eq:lda:LHtransform}
     \llh \mapsto \Trace{ \mathbf{S} \tmat B \mathbf{S}\T (\mathbf{S}\T)^{-1} \tmat W^{-1} \mathbf{S}^{-1} }
     = \Trace{ \mathbf{S} \tmat B \tmat W^{-1} \mathbf{S}^{-1} }
     = \Trace{\tmat B \tmat W^{-1}}
   \end{equation}
   because of the \href{https://en.wikipedia.org/wiki/Trace_(linear_algebra)#Trace_of_a_product}{similarity invariance of the trace}, which follows from its cyclic property \citep[696, C.9]{Bishop:06}: $\Trace{\mathbf{S} \mathbf{A} \mathbf{S}^{-1}} = \Trace{\mathbf{S}^{-1} \mathbf{S} \mathbf{A}} = \Trace{\mathbf{A}}$ \citep[88]{Deisenroth:Faisal:Ong:20}
 \item this means that only the subspace spanned by $\mathbf{A}$ is relevant, not the specific basis implied; we can thus assume without loss of generality that $\mathbf{A}$ is an orthogonal projection, i.e.\ its rows $\va_k\T$ are orthonormal and $\mathbf{A} \mathbf{A}\T = \mathbf{I}_r$
 \item this enables us to simplify the optimisation problem by sphering $\mathbf{W}$ with the same coordinate transformation $\mathbf{S}$ as in Sec.~\ref{sec:lda:standard:sphering}
   \[
     \mathbf{W}' = \mathbf{S} \mathbf{W} \mathbf{S}\T = \mathbf{I},\qquad
     \mathbf{B}' = \mathbf{S} \mathbf{B} \mathbf{S}\T
   \]
 \item using an orthogonal projection $\mathbf{A}'$ from the transformed coordinates to the discriminant space, eq.~(\ref{eq:lda:tildeWB}) becomes
   \begin{equation}
     \label{eq:lda:tildeWBprime}
     \tmat W' = \mathbf{A}' \mathbf{W}' (\mathbf{A}')\T = \mathbf{A}' (\mathbf{A}')\T = \mathbf{I},\qquad
     \tmat B' = \mathbf{A}' \mathbf{B}' (\mathbf{A}')\T
   \end{equation}
   and the $\llh$ statistic is reduced to
   \begin{equation}
     \label{eq:lda:LHprime}
     \llh(\mathbf{A}') = \Trace{\mathbf{A}' \mathbf{B}' (\mathbf{A}')\T}
     = \sum_{k=1}^r (\va_k')\T \mathbf{B}' \va_k'
   \end{equation}
 \item it stands to reason that $\llh(\mathbf{A}')$ is maximised by the first $r$ eigenvectors $\va_k' = \vv_k$ of $\mathbf{B}'$ and corresponding eigenvalues $\mu_k$ \citep[332]{Venables:Ripley:02}, with $\llh(\mathbf{A}') = \sum_{k=1}^r \mu_k$;%
   \footnote{we will not attempt a more formal proof here, but it should be possible to derive optimality of this solution from the \href{https://en.wikipedia.org/wiki/Low-rank_approximation\#Proof_of_Eckart–Young–Mirsky_theorem_(for_Frobenius_norm)}{Eckart-Young-Mirsky} theorem for the Frobenius norm $\Norm[F]{\mathbf{B}'}$, orthogonal decomposition of the Frobenius norm, and the fact that $\Norm[F]{\mathbf{B}'} = \sum_k \mu_k$.}
 \item discriminant axes in the original coordinate system are obtained as in Sec.~\ref{sec:lda:standard:discriminant} by back-transformation $\va_k = \mathbf{S}\T \va_k'$, or in matrix notation $\mathbf{A} = \mathbf{A}' \mathbf{S}$ (since $\va_k\T = (\va_k')\T \mathbf{S}$)
 \item note that $\mathbf{A}$ is usually not an orthogonal projection after the back-transformation, but can be orthogonalised without affecting the $\llh$ criterion because of (\ref{eq:lda:LHtransform})
 \item the same solution is also given by \citet[192]{Bishop:06}; a complete (but very condensed) proof based on direct optimisation of $\llh$ and other separation criteria can be found in \citep[446--452]{Fukunaga:90}
\end{itemize}

%%%%%%%%%%%%%%%%%%%%%%%%%%%%%%%%%%%%%%%%%%%%%%%%%%%%%%%%%%%%%%%%%%%%%%%%
\subsection{Repeated-measures LDA}
\label{sec:lda:repeated}

\begin{itemize}
\item \textbf{repeated-measures} as appropriate terminology: \url{https://en.wikipedia.org/wiki/Repeated_measures_design}
\end{itemize}

%%%%%%%%%%%%%%%%%%%%%%%%%%%%%%%%%%%%%%%%%%%%%%%%%%%%%%%%%%%%%%%%%%%%%%%%
\subsection{Implementation}
\label{sec:lda:implement}

A naive straightforward implementation of LDA consists of the following steps:

\begin{enumerate}
\item Compute between-group variance matrix $\mathbf{B}$ and within-group variance matrix $\mathbf{W}$ according to (\ref{eq:lda:W}) and (\ref{eq:lda:B}).\todo{or corresponding eq.\ for repeated-measures LDA}
\item Determine eigenvalue decomposition $\mathbf{W} = \mathbf{U} \mathbf{D} \mathbf{U}\T$ with $\mathbf{D} = \Diag{\lambda_1, \ldots, \lambda_d}$, checking that $\mathbf{W}$ has full rank and a reasonable condition number, i.e.\ $\lambda_d > \epsilon \lambda_1$ (based on \texttt{tol=}).
\item Construct coordinate transformation $\mathbf{S} = \mathbf{D}^{-\frac12} \mathbf{U}\T$ for sphering $\mathbf{W}$ as welll as its inverse $\mathbf{S}^{-1} = \mathbf{U} \mathbf{D}^{\frac12}$.\todo{Is $\mathbf{S}^{-1}$ really needed?}
\item Compute between-group variance matrix $\mathbf{B}' = \mathbf{S} \mathbf{B} \mathbf{S}\T$ in the new coordinate system.
\item Determine eigenvalue decomposition $\mathbf{B}' = \mathbf{V} \mathbf{E} \mathbf{V}\T$ with $\mathbf{E} = \Diag{\mu_1, \mu_2, \ldots}$.
\item Choose number $r$ of discriminant axes such that $r\leq g-1$, $r\leq \Rank{\mathbf{B}'}$ and $\mu_r > \epsilon \mu_1$ (or perhaps some $R^2$-like criterion).
\item Construct orthogonal discriminant projection $\mathbf{A}' = \mathbf{V}_r\T$, or perhaps $\mathbf{A}' = \mathbf{E}^{-\frac12} \mathbf{V}_r\T$ for equal separation along all discriminant axes; then transform to orginal coordinates $\mathbf{A} = \mathbf{A}' \mathbf{S}$. If the function returns discriminants as column vectors, this can be shortened to $\mathbf{A}\T = \mathbf{S}\T \mathbf{V}_r \bigl[ \mathbf{E}^{-\frac12} \bigr]$.
\item Obtain discriminant scores as $\mathbf{Y} = \mathbf{X} \mathbf{A}\T$.
\end{enumerate}

\todo[inline]{extend sketch to detailed implementation with all equations and algorithms, in particular computing $\mathbf{B}$ and $\mathbf{W}$}

\todo[inline]{improved (?) implementation with SVD of factors of $\mathbf{W}$ and $\mathbf{B}'$}

%%%%%%%%%%%%%%%%%%%%%%%%%%%%%%%%%%%%%%%%%%%%%%%%%%%%%%%%%%%%%%%%%%%%%%%%
%%%%%%%%%%%%%%%%%%%%%%%%%%%%%%%%%%%%%%%%%%%%%%%%%%%%%%%%%%%%%%%%%%%%%%%%
\section{}
\label{sec:A}

%%%%%%%%%%%%%%%%%%%%%%%%%%%%%%%%%%%%%%%%%%%%%%%%%%%%%%%%%%%%%%%%%%%%%%%%
\subsection{}
\label{sec:A:}


%%%%%%%%%%%%%%%%%%%%%%%%%%%%%%%%%%%%%%%%%%%%%%%%%%%%%%%%%%%%%%%%%%%%%%%%
%%%%%%%%%%%%%%%%%%%%%%%%%%%%%%%%%%%%%%%%%%%%%%%%%%%%%%%%%%%%%%%%%%%%%%%%
\section{}
\label{sec:B}

%%%%%%%%%%%%%%%%%%%%%%%%%%%%%%%%%%%%%%%%%%%%%%%%%%%%%%%%%%%%%%%%%%%%%%%%
\subsection{}
\label{sec:B:}


%%%%%%%%%%%%%%%%%%%%%%%%%%%%%%%%%%%%%%%%%%%%%%%%%%%%%%%%%%%%%%%%%%%%%%%%
%%%%%%%%%%%%%%%%%%%%%%%%%%%%%%%%%%%%%%%%%%%%%%%%%%%%%%%%%%%%%%%%%%%%%%%%
%% \renewcommand{\bibsection}{}    % avoid (or change) section heading 
\bibliographystyle{apalike}
\bibliography{stefan-literature,stefan-publications}  

\newpage
\listoftodos

\end{document}
